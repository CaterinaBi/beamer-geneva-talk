\documentclass[lesson_slides]{subfiles}
%\usepackage{natbib}
\usepackage{graphicx}
% \graphicspath{ {./images/} }
\usepackage{enumerate}
\usepackage{pifont} % for ding
\usepackage{float} % keeps tables in the exact position they occupy in the code
\usepackage{xcolor} % text colour
\usepackage{gb4e} % leave last

\begin{document}
%%=-=-=-=-=-=-=-=-=-=-=-=-=-=-=-=-=-=-=-=-=-=-=-=-=-=-=-=-=-=-=-=-=-=-=-=-=-=-=-=
%   FRAME START   -=-=-=-=-=-=-=-=-=-=-=-=-=-=-=-=-=-=-=-=-=-=-=-=-=-=-=-=-=-=-=
\begin{frame}[c]{Our goals}

\transboxin<1>
    \transglitter<2>
    \transwipe<3>
    \noindent \textbf{\textsc{why we redid the study}} \pause 
    \begin{itemize}
        \item[\ding{227}] replicate Larrivé’s study in a more thorough (and syntax-informed) way: \pause
            \begin{itemize}
            \item extended to the totality of the corpora (ESLO 1, ESLO2); \pause
            \item extended to more wh-words: comment ‘how’, quand ‘when’, où ‘where’ qui-object ‘who’ and quoi-object ‘what’; \pause
            \item 3 activation levels; \pause
            \item extracted and cleaned the data differently. \pause
            \end{itemize}
        \item[\ding{227}] evaluate the results against the controversy discussed earlier.
        \end{itemize} 
  
\end{frame}
%   FRAME END   --==-=-=-=-=-=-=-=-=-=-=-=-=-=-=-=-=-=-=-=-=-=-=-=-=-=-=-=-=-=-=

\begin{frame}{Our results (in a nutshell)}

\transboxin<1>
    \transglitter<2>
    \transwipe<3>
    \noindent \textbf{\textsc{our results}} \pause 
    \begin{itemize}
        \item[\ding{227}] in our corpora, wh-in situ has no categorical pragmatic value; \pause
        \item[\ding{227}] the majority in both corpora is populated by a third type of wh-in situ non considered by Larrivée; \pause
        \item[\ding{227}] this type is followed in the second corpus non-activated wh-in situ; \pause
        \item[\ding{227}] explicit activation is rare; \pause
        \item[\ding{227}] our results differ from Larrivée’s, except for the trends seen for non-activation.
    \end{itemize}
    
\end{frame}

\end{document}

