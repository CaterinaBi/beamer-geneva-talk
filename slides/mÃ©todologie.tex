\documentclass[lesson_slides]{subfiles}
%\usepackage{natbib}
\usepackage{graphicx}
% \graphicspath{ {./images/} }
\usepackage{enumerate}
\usepackage{pifont} % for ding
\usepackage{float} % keeps tables in the exact position they occupy in the code
\usepackage{xcolor} % text colour
\usepackage{gb4e} % leave last

\begin{document}
%%=-=-=-=-=-=-=-=-=-=-=-=-=-=-=-=-=-=-=-=-=-=-=-=-=-=-=-=-=-=-=-=-=-=-=-=-=-=-=-=
%   FRAME START   -=-=-=-=-=-=-=-=-=-=-=-=-=-=-=-=-=-=-=-=-=-=-=-=-=-=-=-=-=-=-=
\begin{frame}[c]{The corpora}

    \transboxin<1>
    \transglitter<2>
    \transwipe<3>
    Our hypothesis was tested utilising corpus linguistics techniques. \pause

    \noindent \textbf{\textsc{notre hypothèse de travail}} \pause
    \begin{itemize}
        \item[\ding{227}] les langues n'évoluent pas seulement dans la direction de non-mouvement phrasal (IM$=$0); \pause
        \item[\ding{227}] elles évoluent aussi vers la direction de non-mouvement de tête (IM\textsubscript{lex}$=$0) et non-spellout (Spellout$=$0) 
    \end{itemize}
    
\end{frame}
%   FRAME END   --==-=-=-=-=-=-=-=-=-=-=-=-=-=-=-=-=-=-=-=-=-=-=-=-=-=-=-=-=-=-=
%   FRAME START   -=-=-=-=-=-=-=-=-=-=-=-=-=-=-=-=-=-=-=-=-=-=-=-=-=-=-=-=-=-=-=
\begin{frame}[c]{The corpora}

    \transboxin<1>
    \transglitter<2>
    %\transwipe<3>
    \noindent The corpora used for this part of the study are: \pause

    \begin{itemize}
        \item[\ding{227}] \vspace*{-2mm} ESLO 1 and ESLO 2 (spoken French, 1970-2014) (data from Baunaz \& Bonan (in prep.)); \pause
        \item[\ding{227}] \vspace*{-2mm} Bonan's corpus of 485 theatre scripts (late 1500-early 1900).
    \end{itemize}
    
\end{frame}
%   FRAME END   --==-=-=-=-=-=-=-=-=-=-=-=-=-=-=-=-=-=-=-=-=-=-=-=-=-=-=-=-=-=-=
%   FRAME START   -=-=-=-=-=-=-=-=-=-=-=-=-=-=-=-=-=-=-=-=-=-=-=-=-=-=-=-=-=-=-=
\begin{frame}[c]{Data classification}

    \transboxin<1>
    \transglitter<2>
    \transwipe<3>
    \noindent The data were pre-classified automatically (and then checked manually). \pause

    \textbf{\textsc{basic classification}} \pause
    \begin{enumerate}
        \item position of the wh-element \pause (ex situ, in situ); \pause
        \item interrogative strategy \pause (SV, est-ce que, VS); \pause
        \item well-formedness \pause (formed, fragment). \pause
    \end{enumerate}
    
\end{frame}
%   FRAME END   --==-=-=-=-=-=-=-=-=-=-=-=-=-=-=-=-=-=-=-=-=-=-=-=-=-=-=-=-=-=-=
%   FRAME START   -=-=-=-=-=-=-=-=-=-=-=-=-=-=-=-=-=-=-=-=-=-=-=-=-=-=-=-=-=-=-=
\begin{frame}[c]{Data cleaning}

    \transboxin<1>
    \transglitter<2>
    \transwipe<3>
    \textbf{\textsc{data that we considered}} \pause
    
    \begin{itemize}
        \item[\ding{227}] wh-questions \pause (as opposed to yes/no questions); \pause
        \item[\ding{227}] matrix monoclausal questions \pause (no long distance questions, no embedded questions, no infinitives, etc.); \pause
        \item[\ding{227}] real questions \pause (no rethoric questions, echo questions, quiz questions, introspective questions etc.); \pause
        \item[\ding{227}] well-formed clauses \pause (no fragments, no questions containing only a wh-element ('Qui?', 'Quoi?')).
    \end{itemize}
    
\end{frame}
%   FRAME END   --==-=-=-=-=-=-=-=-=-=-=-=-=-=-=-=-=-=-=-=-=-=-=-=-=-=-=-=-=-=-=
\begin{frame}[c]{Data cleaning}

    \transboxin<1>
    \transglitter<2>
    \transwipe<3>
    \textbf{\textsc{which wh-elements?}} \pause
    
    \begin{itemize}
        \item[\ding{227}] we only studied non-lexically restricted wh-elements ('what?', no 'which chair?'); \pause
        \item[\ding{227}] today, I shall only present the data on those wh-elements that can surface either in situ or ex situ (e.g. not 'que'): \pause
        \begin{enumerate}
            \item quand; \pause
            \item où; \pause
            \item comment; \pause
            \item quoiO; \pause
            \item quiO.
        \end{enumerate}
    \end{itemize}
    
\end{frame}
%=-=-=-=-=-=-=-=-=-=-=-=-=-=-=-=-=-=-=-=-=-=-=-=-=-=-=-=-=-=-=-=-=-=-=-=-=-=-=-=
\end{document}